%% If you prefer -- and have been allowed -- to use
%% Arial, then
%% \documentclass[arial]{usmthesis}
%% It's not really Arial, it's a Helvetica look-alike,
%% but if you're not a designer nor a typographer, you
%% probably can't tell the difference (I can't either)
\documentclass{usmthesis}
%%%%%%%%%%%%%%%%%%%%%%%%%%%%%%%%%%%%%%%%%%%%%%%%%%%%%%%
% This is usmthesis.tex, 5 October 2016.
% Created by Lim Lian Tze (Ph.D.)
% liantze@gmail.com
% http://liantze.penguinattack.org.
%
% This is the "main" file for the thesis,
% formatted according to the Guide to the
% Preparation, Submission and Examination of
% Theses, published by IPS USM.
%%%%%%%%%%%%%%%%%%%%%%%%%%%%%%%%%%%%%%%%%%%%%%%%%%%%%%%

%% Example of loading other packages that you may require.
%% I'm loading the marvosym package so that I can produce a
%% smiley face with the command \Smiley.
\usepackage{marvosym}

%% Also, the enumitem package is great for customising
%% list environments.
\usepackage{enumitem}

%% Listings is a nice package for typesetting code
%% listings. Other possible packages include fancyvrb,
%% minted, etc.
\usepackage{listings}
\lstset{basicstyle=\ttfamily,breaklines=true}

%% For those who need to produce algorithms and pseudocode.
%% There are a number of different packages available, but
%% unfortunately they tend not to work well together!
%% I'm using algorithmicx, specifically algpseucode, here.
\usepackage{algpseudocode}
\usepackage{algorithm}
\usepackage{indentfirst}

%% Enter particulars about your thesis HERE
% Your Name
\author{Lim Lian Tze}
% English title of your thesis
\title{Writing Your Thesis with LaTeX with a Very, Very, Very Long Title}
% Malay title of your thesis
\titlems{Penulisan Tesis dengan LaTeX}
% Year submitted
\submityear{2015}
% Month submitted
\submitmonth{December}
%% Choose only 1 degree type! :-)
\degreetype{Doctor of Philosphy}
% \degreetype{Master of Science}


%%%%%%%%%%%%%%%%%%%%%%%%%%%%%%%%%%%%%%%%%%%%%%%%%%%%%%%
%  You can comment out the following line if you don't have a
% "List of Own Publications"
%%%%%%%%%%%%%%%%%%%%%%%%%%%%%%%%%%%%%%%%%%%%%%%%%%%%%%%
\newcites{own}{List of Publications}

\usepackage[plainpages=false,bookmarksnumbered,breaklinks=true]{hyperref}
%%%%%%%%%%%%%%%%%%%%%%%%%%%%%%%%%%%%%%%%%%%%%%%%%%%%%%%
% Options for generating hyperlinks when using pdfLaTeX
%%%%%%%%%%%%%%%%%%%%%%%%%%%%%%%%%%%%%%%%%%%%%%%%%%%%%%%
\ifpdf
  \makeatletter
  \hypersetup{hypertexnames=false,%
    pdfauthor={\@author},pdftitle={\@title}}
  \makeatother
\fi

\usepackage{ragged2e}
\begin{document}

%%%%%%%%%%%%%%%%%%%%%%%%%%%%%%%%%%%%%%%%%%%%%%%%%%%%%%%
% Default bibliography style is apa (using 
% \RequirePackage[natbibapa]{apacite} in the class file).
%
% If you prefer the number system though, use bibliography
% style "plainnat" for [1][2][3] or "alpha" for [Jon94] (the label
% will be auto-generated).
%%%%%%%%%%%%%%%%%%%%%%%%%%%%%%%%%%%%%%%%%%%%%%%%%%%%%%%
\bibliographystyle{apacite}
\bibliographystyleown{apacite}
%\bibliographystyle{plainnat}
%\bibliographystyleown{plainnat}

\frontmatter

%%%%%%%%%%%%%%%%%%%%%%%%%%%%%%%%%%%%%%%%%%%%%%%%%%%%%%%
% Inserts the cover page (the hard cover with gold-lettering)
% and the title page 
%%%%%%%%%%%%%%%%%%%%%%%%%%%%%%%%%%%%%%%%%%%%%%%%%%%%%%%
\makecover


%%%%%%%%%%%%%%%%%%%%%%%%%%%%%%%%%%%%%%%%%%%%%%%%%%%%%%%
% MAKE SURE YOU HAVE A acknowledgements.tex FILE
%%%%%%%%%%%%%%%%%%%%%%%%%%%%%%%%%%%%%%%%%%%%%%%%%%%%%%%
\chapter{Acknowledgements}

Many thanks to Prof.~Donald Knuth for giving us \TeX, and Leslie Lamport for \LaTeX.  

Dr.~Dhanesh's effort in officially raising awareness about \LaTeX{} at USM during NaCSPC'05 was incredibly refreshing and much welcome. Dr.~Dhanesh also followed this up by suggesting that I give an introductory workshop on \LaTeX{} during CSPC'07, and to \LaTeX{} the programme book cum abstract collections (a first!).  Many, many thanks to him and Dr.~Azman, who was \emph{very} supportive during the last minute changes and glitches.  Also, thanks to Nur Hussein, Adib, Seng Soon and Anusha for happily trying out early versions of my \LaTeX{} thesis templates, and for their feedbacks, and to everyone who attended my talk, requested tutorials, downloaded from my website (\url{http://liantze.googlepages.com/latextypesetting}), etc, etc.

Hope everyone graduates quickly then!

%Thanks to Dr.~Donald Knuth for giving us \TeX, Leslie Lamport for \LaTeX, the NaCSPC'05 committee and Dr.~Dhanesh for spreading the \LaTeX word and material.

%Hope everyone graduates quickly!

\tableofcontents \clearpage
\listoftables \clearpage
\listoffigures \clearpage
%%%%%%%%%%%%%%%%%%%%%%%%%%%%%%%%%%%%%%%%%%%%%%%%%%%%%%%
% You can comment out the following line if you don't
% have a "List of Plates"
%%%%%%%%%%%%%%%%%%%%%%%%%%%%%%%%%%%%%%%%%%%%%%%%%%%%%%%
\listofplates \clearpage

%%%%%%%%%%%%%%%%%%%%%%%%%%%%%%%%%%%%%%%%%%%%%%%%%%%%%%%
% You can comment out the following line if you don't
% have a "List of Acronyms"
%%%%%%%%%%%%%%%%%%%%%%%%%%%%%%%%%%%%%%%%%%%%%%%%%%%%%%%
\chapter{List of Abbreviations}

\begin{acronym}[UTMK] %% replace 'MMMM' with the longest acronym in your list
\setlength{\itemsep}{0pt}
\acro{IPS}{Institut Pengajian Siswazah}
\acro{PPSK}{Pusat Pengajian Sains Komputer}
\acro{USM}{Universiti Sains Malaysia}
\acro{UTMK}{Unit Terjemahan Melalui Komputer}
\end{acronym}

\chapter{List of Symbols}

\begin{acronym}[lim ]
\setlength{\itemsep}{0pt}
\acro{lim}[$\lim{}$]{limit}
\acro{theta}[$\theta{}$]{angle in radians}
\end{acronym}


% Paragraph spacing
\setlength\parskip{18pt}
% Text-float spacing
\setlength\intextsep{24pt}

%%%%%%%%%%%%%%%%%%%%%%%%%%%%%%%%%%%%%%%%%%%%%%%%%%%%%%%
% Your Malay and English abstracts, each in one file.
%%%%%%%%%%%%%%%%%%%%%%%%%%%%%%%%%%%%%%%%%%%%%%%%%%%%%%%
\input{abs-mal}
\input{abs-eng}

 
\mainmatter

%%%%%%%%%%%%%%%%%%%%%%%%%%%%%%%%%%%%%%%%%%%%%%%%%%%%%%%
% The actual chapters of your thesis as listed in 
% mainchaps.tex. Make sure you have the relevant
% chapter files. 
% E.g. if you mainchaps.tex contains the lines
%
%  \include{hypothesis.tex}
%  \include{proof.tex}
%
% Then you MUST have the files hypothesis.tex, proof.tex
% (containing the relevant chapters) in the same directory
% as mainchaps.tex.
%%%%%%%%%%%%%%%%%%%%%%%%%%%%%%%%%%%%%%%%%%%%%%%%%%%%%%%
\input{chap-intro}  % note need to use \input here for \addtocontents{toc}... to work
\include{chap-review}
\include{chap-design}
\include{chap-implementation}
\include{chap-discussion}
\include{chap-conclusion}

%%%%%%%%%%%%%%%%%%%%%%%%%%%%%%%%%%%%%%%%%%%%%%%%%%%%%%%
% The bibliography. Turn on page numbering.
%%%%%%%%%%%%%%%%%%%%%%%%%%%%%%%%%%%%%%%%%%%%%%%%%%%%%%%
\addtocontents{toc}{\protect\cftpagenumberson{chap}}
\bibliography{mybib}


%%%%%%%%%%%%%%%%%%%%%%%%%%%%%%%%%%%%%%%%%%%%%%%%%%%%%%%
% The appendices.
% If you don't have any, you may delete everything below,
% until and including \input{appendices}.
%%%%%%%%%%%%%%%%%%%%%%%%%%%%%%%%%%%%%%%%%%%%%%%%%%%%%%%
\appendix
\assignpagestyle{\chapter}{empty}
%% * If IPS says they don't want any page numbering in the footer,
%%   add \pagestyle{empty}
%% * If they don't want any page numbering in the ToC either,
%%   add \addtocontents{toc}{\protect\cftpagenumbersoff{chap}}
%% * If they say they don't want Appendix A, B, C... to appear
%%   in the ToC either, add
%%     \addtocontents{toc}{\protect\setcounter{tocdepth}{-1}}
%%     \addtocontents{toc}{\texbf{List of Publications}} % (to get it to appear)
\input{appendices}
\assignpagestyle{\chapter}{plain}

%%%%%%%%%%%%%%%%%%%%%%%%%%%%%%%%%%%%%%%%%%%%%%%%%%%%%%%
% The list of own publications.  If you don't have one, you may
% comment out the next 4 lines.
%%%%%%%%%%%%%%%%%%%%%%%%%%%%%%%%%%%%%%%%%%%%%%%%%%%%%%%
\nociteown{lim:2007,lim:latextypesetting}
\bibliographyown{mybib}

\end{document}
